\documentclass[a4paper]{article}
\usepackage[utf8]{inputenc}
\usepackage[T1]{fontenc}
\usepackage{textcomp}
\usepackage[czech]{babel}
\usepackage{amsmath, amsthm, amssymb}


% figure support
\usepackage{import}
\usepackage{xifthen}
\pdfminorversion=7
\usepackage{pdfpages}
\usepackage{transparent}
\newcommand{\incfig}[1]{%
    \def\svgwidth{\columnwidth}
    \import{./figures/}{#1.pdf_tex}
}

\theoremstyle{definition}
\newtheorem{definition}{Definice}[section]

\pdfsuppresswarningpagegroup=1
\author{Hynek Kydlicek}
\title{Principy počitáčů}
\begin{document}
\maketitle
\section{6. Přednáška}
\subsection{Paměti}
% Paměť je posloupnost bytů pro ukládání dat.

% Jednotlivé byty je třeba adresovat.
\begin{definition}[Adresový prostor]
    Adresový prostor definuje, kolika byty můžeme adresovat.
    Adresový prostor $\neq $ kapacity paměti, všechny adresy totiž nemusejí být využité
\end{definition}
\begin{definition}[Slovo, en:WORD]
    Slovo v informatice znamená přenosovou jednotku paměti.
    Tedy kolik bytů můžeme přenést jednou transakcí.
    V pejorativním významu můžeme slovo brát jako označení pamětového bloku o velikosti 2B.
    \\
    DWORD můžeme brát jako označení pro 3B bloku.
    \\
    QWORD jako 4B bloku.
\end{definition}

\subsection{Jednotky SI a jiné}%
Co znamená 1 KB ?
\begin{enumerate}
    \item $1000B$ (dle výrobců disků, je to méně)
    \item $1024B$ = $2^{10}B$ standard
\end{enumerate}
Důsledek = zavedení nové jednotky $KiB = 1024B$, moc se ale neuživá.
\\
Podobně pro další jednotky.
\begin{enumerate}
    \item MB = $1024KB = 2^20B$
    \item GB = $1024MB = 2^30B$
\end{enumerate}
\subsection{SRAM vs DRAM}%
\begin{definition}[RAM, RANDOM ACCESS MEMORY]
    RAM je paměť, pro kterou platí, že přístup na každý prvek trvá stejně rychle.
    Není úplně pravda díky cache v procesoru je sekvenční přístup rychlejší.
\end{definition}
\begin{definition}[Volatile Memory]
    Volatile memory je paměť, která udrží data pouze pokud je připojena k elektřině.
    Po vypnutí ztrácí veškerý svůj obsah.
\end{definition}

\begin{center}
\begin{tabular}{c c}
    SRAM & DRAM\\
    1 & 2
\end{tabular}
\end{center}
\end{document}
