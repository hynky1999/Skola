\documentclass[a4paper]{article}

\usepackage[utf8]{inputenc}
\usepackage[T1]{fontenc}
\usepackage{textcomp}
\usepackage[czech]{babel}
\usepackage{amsmath, amssymb}


% figure support
\usepackage{import}
\usepackage{xifthen}
\pdfminorversion=7
\usepackage{pdfpages}
\usepackage{transparent}
\newcommand{\incfig}[1]{%
    \def\svgwidth{\columnwidth}
    \import{./figures/}{#1.pdf_tex}
}
\pdfsuppresswarningpagegroup=1
\author{Hynek Kydlíček}
\title{Diskrétka 5}
\begin{document}
\maketitle
\section{Příklad 1}
Výsledek (Počet obědů) = $\mid\bigcup_{i=1}^5 A_i\mid$, kde $A_i$ je i-tý známý.
\\
Použijeme princip inkluze a exkluze.
\begin{align*}
    \mid\bigcup_{i=1}^5 A_i\mid &= {5\choose1}*10 - {5\choose2}*5 + {5\choose3}*3 - {5\choose4}*2 + {5\choose5}*1\\
    &= 50 - 50 + 30 - 10 + 1\\
    &= 21
\end{align*}
Tedy obědval 21 krát. Pokud se nestalo, že by za jeden den stihl dva obědy, tak program trval 21 dní.

\section{Příklad 2}
Součet oprávněných voličů = $\mid\bigcup_{i=1}^3 A_i\mid + n$, kde $A_1$ = voliči Alice, $A_2$ = voliči Boba,
$A_3$ = voliči Charlieho, $n$ = počet nevoličů. Samozřejmě můžeme místo počtu voličů počítat s procenty.
\\
Použijeme princip inkluze a exkluze.
\begin{align*}
    \mid\bigcup_{i=1}^3 A_i\mid = 65 + 57 + 58 - 28 - 30 - 27 + 12 = 107\%
\end{align*}
Tedy nevoličů by muselo být -7\%, což je nemožné.

\section{Příklad 3}
Na výpočet použijeme Princip inkluze a exkluze a známý vzorec pro součet aritmetické posloupnosti.
\begin{align*}
    \text{Součet} &= \sum (\bigcup_{i\in \{2,3,7\}} \{x \mid i|x \wedge x \in [4200]\})\\
                  &= \frac{(2+4200) * 2100}{2} + \frac{(3+4200) * 1400}{2} + \frac{(7+4200)*600}{2} \\
                  &- \frac{(6+4200)*700}{2} - \frac{(14+4200)*300)}{2} - \frac{(21+4200)*200}{2} + \frac{(42+4200)*100}{2}\\
                  &= \frac{4200*3000+4200}{2} = 2100*3000 + 2100 = 6300000 + 2100 = 6302100
\end{align*}
\end{document}
