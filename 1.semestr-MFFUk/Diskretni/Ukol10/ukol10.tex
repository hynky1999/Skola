\documentclass[a4paper]{article}

\usepackage[utf8]{inputenc}
\usepackage[T1]{fontenc}
\usepackage{textcomp}
\usepackage[czech]{babel}
\usepackage{amsmath, amssymb}


% figure support
\usepackage{import}
\usepackage{xifthen}
\pdfminorversion=7
\usepackage{pdfpages}
\usepackage{transparent}
\newcommand{\incfig}[1]{%
    \def\svgwidth{\columnwidth}
    \import{./figures/}{#1.pdf_tex}
}
\pdfsuppresswarningpagegroup=1
\begin{document}
\section{Úkol 1}
Stačí ukázat jedno rovinné nakreslení.
\begin{figure}[ht]
    \centering
    \incfig{rov-graf}
    \caption{Nakreslení rov. grafu}
    \label{fig:rov-graf}
\end{figure}
\section{Úkol 2}
    \begin{subsection}{Důkaz dle Kurakowského věty}
        Nalezneme dělení podgrafu $K_{3,3}$ viz \ref{fig:Kurakovského věta}. 
        Zelená označuje jednu partitu, červená druhou partitu. Červené hrany v podgrafu zůstanou, černé ne.
\begin{figure}[ht]
    \centering
    \incfig{kurakow}
    \caption{Kurakowského věta}
    \label{fig:Kurakovského věta}
\end{figure}
    \end{subsection}
    \pagebreak
    \begin{subsection}{Důkaz dle nerovinnosti $K_5$}
        Stačí kontrahovat zelené hrany jako na obrázku \ref{fig:k5}.
\begin{figure}[ht]
    \centering
    \incfig{k5}
    \caption{Neorvinnost $K_5$}
    \label{fig:k5}
\end{figure}
        
    \end{subsection}
\end{document}
