\documentclass[a4paper]{article}

\usepackage[utf8]{inputenc}
\usepackage[T1]{fontenc}
\usepackage{textcomp}
\usepackage[czech]{babel}
\usepackage{amsmath, amssymb, amsthm}


% figure support
\usepackage{import}
\usepackage{xifthen}
\pdfminorversion=7
\usepackage{pdfpages}
\usepackage{transparent}
\newcommand{\incfig}[1]{%
    \def\svgwidth{\columnwidth}
    \import{./figures/}{#1.pdf_tex}
}

\newtheorem{theorem}{Věta}[section]
\newtheorem{lemma}[theorem]{Lemma}
\swapnumbers
\theoremstyle{definition}
\newtheorem{definition}{Definice}[section]

\pdfsuppresswarningpagegroup=1
\author{Hynek Kydlicek}
\title{Diskrétka}
\begin{document}
\maketitle
\section{Grafy}
\begin{definition}[Eulerovský tah]
    Tah, který vede přes všechny hrany grafu právě jednou.
\end{definition}
\begin{definition}[Eulerovský graf]
    Graf je eulerovský $\iff$ existuje v něm uzavřený eulerovský tah.
\end{definition}
\begin{theorem}[Alternativní definice eulerovské grafu]
    Graf je eulerovský $\iff$ graf je souvislý a stupeň každého vrcholu je sudý.
\end{theorem}
\begin{figure}[ht]
    \centering
    \incfig{ahoj}
    \caption{ahoj}
    \label{fig:ahoj}
\end{figure}
\end{document}
