\documentclass[a4paper]{article}

\usepackage[utf8]{inputenc}
\usepackage[T1]{fontenc}
\usepackage{textcomp}
\usepackage[czech]{babel}
\usepackage{amsmath, amssymb}


% figure support
\usepackage{import}
\usepackage{xifthen}
\pdfminorversion=7
\usepackage{pdfpages}
\usepackage{transparent}
\newcommand{\incfig}[1]{%
    \def\svgwidth{\columnwidth}
    \import{./figures/}{#1.pdf_tex}
}

\pdfsuppresswarningpagegroup=1

\begin{document}
\begin{section}{Dcv. 1}
    Matici upravíme do RREF tvaru, abychome našli názi $R(A)$ a $S(A)$.
    \[
        \begin{pmatrix} 1 && 2 && 3 \\
                        -1 && 4 && 1 \\
                        -1 && 1 && -1
        \end{pmatrix}   
        \sim
        \begin{pmatrix} 1 && 2 && 3 \\
                        0 && 6 && 4 \\
                        0 && 3 && 2
        \end{pmatrix}   
        \sim
        \begin{pmatrix} 1 && 2 && 3 \\
                        0 && 6 && 4 \\
                        0 && 0 && 0
        \end{pmatrix}   
    .\] 
    Báze $R(A)$ je $\begin{pmatrix} 1 \\ 2 \\ 3\end{pmatrix}, \begin{pmatrix} 0 \\ 6 \\ 4 \end{pmatrix}$.\\
            Báze $S(A)$ je $\begin{pmatrix} 1 \\ -1 \\ -1 \end{pmatrix}, \begin{pmatrix} 2 \\ 4 \\1 \end{pmatrix}$.
            Abychom zjistil průnik ptáme se, kdy platí rovnice: \[
            a*\begin{pmatrix} 1 \\ 2 \\ 3\end{pmatrix} + b*\begin{pmatrix} 0 \\ 6 \\ 4 \end{pmatrix} = c*\begin{pmatrix} 1 \\ -1 \\ -1 \end{pmatrix} + d*\begin{pmatrix} 2 \\ 4 \\1 \end{pmatrix}
            .\] 
            \[
            \forall a,b,c,d \in R
            .\] 
            Odečteme pravou stranu od levé a přepíšeme do matice.
            \begin{align*}
                &\begin{pmatrix}
                    1 & 0 & -1 & -2 & \bigm| & 0 \\
                    2 & 6 & 1 & -4 & \bigm| & 0 \\
                    3 & 4 & 1 & -1 & \bigm| & 0
                \end{pmatrix}
                \sim
                \begin{pmatrix}
                    1 & 0 & -1 & -2 & \bigm| & 0 \\
                    0 & 6 & 3 & 0 & \bigm| & 0 \\
                    0 & 4 & 4 & 5 & \bigm| & 0
                \end{pmatrix}
                \\
                \sim
                &\begin{pmatrix}
                    1 & 0 & -1 & -2 & \bigm| & 0 \\
                    0 & 6 & 3 & 0 & \bigm| & 0 \\
                    0 & 0 & 12 & 30 & \bigm| & 0
                \end{pmatrix}
                \sim
                \begin{pmatrix}
                    1 & 0 & -1 & -2 & \bigm| & 0 \\
                    0 & 2 & 1 & 0 & \bigm| & 0 \\
                    0 & 0 & 2 & 5 & \bigm| & 0
                \end{pmatrix}
            \end{align*}
            Řešením matice je \[
                \begin{pmatrix} -\frac{d}{2} \\ \frac{5d}{4} \\ -\frac{5d}{2} \\ d  \end{pmatrix} 
            .\] 
            Dosadíme řešení do rovnice: \[
                \frac{-d}{2}*\begin{pmatrix} 1 \\ 2 \\ 3\end{pmatrix} + \frac{5d}{4}*\begin{pmatrix} 0 \\ 6 \\ 4 \end{pmatrix} = \frac{-5d}{2}*\begin{pmatrix} 1 \\ -1 \\ -1 \end{pmatrix} + d*\begin{pmatrix} 2 \\ 4 \\1 \end{pmatrix}
            .\] 
            A upravíme(upravujeme pouze jednotlivé strany !!) \[
                \frac{d}{2}*\begin{pmatrix} -1 \\ 13 \\ 7\end{pmatrix} = \frac{d}{2}*\begin{pmatrix} -1 \\ 13 \\ 7 \end{pmatrix}
            .\] 
            Kromě toho, že jsme si ověřili, že nám řešení vyšlo dobře, tak vidíme, že průnik musí být tvaru \[
            x \in \left<\begin{pmatrix} -1 \\ 13 \\ 7 \end{pmatrix}\right> 
            .\] 
            Tedy tento vektor je bazí průniku(určitě je LN).
\end{section}
\begin{section}{Dcv. 2}
    Z přednášky víme, že \[
        rank(X) + dim(Ker(X)) = n
    \] \[
    dim(S(X)) = dim(R(X)) = dim(S(X^T)) = rank(X) \text{Pro X} \in R^{m,n}
    .\]      První rovnost upravíme na
    \[
        dim(Ker(X)) = n - rank(X)
    .\] 
        Dosazením A za X dostáváme
        \[
            dim(Ker(A)) = n - rank(A)
        .\] 
        Dosazením $A^T$ za X dostáváme
        \[
            dim(Ker(A^T)) = n - rank(A^T) = n - S(A^T) = n - R(A) = n - rank(A)
        .\] 
        Spojením rovnic dostaneme
        \[
            dim(Ker(A)) = n - rank(A) = dim(Ker(A))
        .\] 
\end{section}

\end{document}
