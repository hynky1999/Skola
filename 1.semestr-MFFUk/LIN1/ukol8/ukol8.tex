\documentclass[a4paper]{article}

\usepackage[utf8]{inputenc}
\usepackage[T1]{fontenc}
\usepackage{textcomp}
\usepackage[czech]{babel}
\usepackage{amsmath, amssymb}


% figure support
\usepackage{import}
\usepackage{xifthen}
\pdfminorversion=7
\usepackage{pdfpages}
\usepackage{transparent}
\newcommand{\incfig}[1]{%
    \def\svgwidth{\columnwidth}
    \import{./figures/}{#1.pdf_tex}
}

\pdfsuppresswarningpagegroup=1
\title{LIN úkol 8}
\author{Hynek Kydlíček}
\begin{document}
\maketitle
\section{Dcv. 1}
Postupujme sporem, předpokládejme, že existují alespoň 2 řešení soustavy a zárověň jsou sloupce LN.
Existují 2 řešení($r_1$, $r_2$) říká, že existují 2 různé lin. kombiace sloupců dávají vektor $b$.
Tedy
\[
    \sum_{i=1}^{n} r_2_i * s_i = b
.\] 
a
\[
    \sum_{i=1}^{n} r_1_i * s_i = b
.\] 
A tedy
\begin{align*}
    \sum_{i=1}^{n} r_2_i * s_i  = \sum_{i=1}^{n} r_1_i * s_i\\
    \sum_{i=1}^{n} (r_2_i-r_1_i) * s_i  = 0
\end{align}
Avšak $r_2 \neq r_1 \implies (\exists i \in n)\ r_1_i \neq  r_2_i$.
To je ale spor s LN sloupců, protože jsme právě našli netriviální lin. kombinaci dávající nulový vektor.


\section{Dcv. 2}
\subsection{Implikace zleva doprava}
$v_1\ldots v_n \in V$ LN $\implies v_1,v_1+v_2,\sum_{i\in [n]} v_i$ jsou LN.\\
Ukážeme, že linearní kombinace vektorů $v_1,v_1+v_2,\sum_{i\in [n]} v_i$ rovnající se $0$ musí být triviální.
\\

\[
    \sum_{i=1}^{n}(\sum_{k=1}^{i}v_i * \alpha_i) = 0\ \text{lze přepsat jako} \sum_{i=1}^{n}(v_i*\sum_{k=i}^{n} \alpha_k) = 0
.\] 
Z předpokladu:
\[
    v_1\ldots v_n\ \text{jsou LN} \implies \sum_{i=1}^{n} \beta_i*v_i = 0\  \text{pouze pro}\  \beta_i = 0
.\]
Tedy pouze pokud $\sum_{k=i}^{n} \alpha_k = \beta_i = 0$ pro každe $i \in n$ pak je lin. kombinace $= 0$. Pokud bychom si takovoto soustavu zapsali do matice 
\[
    A^{n,n}=
\left(
\begin{array}{c c c c | c}
    1 & 1 & \ldots & 1 & 0\\
    0 & 1 & \ldots & 1 & 0\\
    \vdots & \vdots & \ddots & \vdots & 0\\
    0 & 0 & 0 & 1 & 0
\end{array} 
\right)
.\]

Vidíme, že matice j́e v REF a má právě jedno řešení(Máme n pivotů a n řadků neboli matice je regularní) $(\forall i \in n)  \alpha_i = 0$. A tedy pouze triviální kombinace vektorů je rovna nule. A tedy vektory musí být LN.

\subsection{implikace zprava doleva}
Dokážeme obrácenou implikaci.
$v_1\ldots v_n \in V$ jsou LZ $\implies v_1,v_1+v_2,\sum_{i\in [n]} v_i$ jsou LZ.\\
Ukážeme, že existuje netriviální linearní kombinace vektorů $v_1,v_2\ldots v_n$ rovnající se $0$.
\[
    \sum_{i=1}^{n}(\sum_{k=1}^{i}v_i * \alpha_i) = 0\ \text{lze přepsat jako} \sum_{i=1}^{n}(v_i*\sum_{k=i}^{n} \alpha_k) = 0
.\] 
Z předpokladu platí, že:
\[
    v_1\ldots v_n\ \text{jsou LZ} \implies \sum_{i=1}^{n} \beta_i*v_i = 0\  \text{existuje alespoň jedno i,}\  \beta_i \neq 0
.\]
Tedy pokud $\sum_{k=i}^{n} \alpha_k = \beta_i$, pro každe $i\in n$ pak lin. kombinace $=0$. Zapíšeme soustavu jako matici
\[
    A^{n,n}=
\left(
\begin{array}{c c c c | c}
    1 & 1 & \ldots & 1 & \beta_1\\
    0 & 1 & \ldots & 1 & \beta_2\\
    \vdots & \vdots & \ddots & \vdots & \beta_3\\
    0 & 0 & 0 & 1 & \beta_n
\end{array} 
\right) \text{, kde alespň jedno}\  \beta_i \neq 0
.\] 
Vidíme, že matice j́e v REF a má právě jedno řešení(znovu máme regularní matici),takové řešení určite nebude $(\forall i \in n)  \alpha_i = 0$. To by znamenalo, že $(\forall i \in n) \beta_i = 0$. Což není pravda. Tedy alespoň jedno $\alpha_i \neq  0$. A tedy existuje netriviální kombinace vektorů rovna nule. A tedy vektory musí být LZ.
\section{Dcv. 1}
Úloha můžeme formulovat také tak, kdy má rovnice
\begin{equation*}
\alpha_1 * \begin{pmatrix} 1\\ a\\ 1 \end{pmatrix}\
+ 
\alpha_2 * \begin{pmatrix} 1\\ 1\\ 1 \end{pmatrix}\
*
\alpha_3 * \begin{pmatrix} 2\\ 2\\ a \end{pmatrix}\
= 0
\end{equation}
právě jedno řešení(triviální).
Soustavu si přepíšeme do matice.
\begin{equation*}
    \left(
    \begin{array}{c c c | c}
        1 & 1 & 2 & 0\\
        a & 1 & 2 & 0\\
        1 & 1 & a & 0
    \end{array}
    \right)
    \sim
    \left(
    \begin{array}{c c c | c}
        1 & 1 & 2 & 0\\
        0 & 1-a & 2-2*a & 0\\
        0 & 0 & a-2 & 0
    \end{array}
    \right)
\end{equation}
Pro a=2 dostáváme matici
\begin{equation}
\left(
\begin{array}{c c c | c}
    1 & 1 & 2 & 0\\
    0 & -3 & -2 & 0\\
    0 & 0 & 0 & 0
\end{array}
\right)
\end{equation}
Matice má určitě více než jedno řešení(má jeden sloupec nebazický).
\\
Pro a=1
\begin{equation}
\left(
\begin{array}{c c c | c}
    1 & 1 & 2 & 0\\
    0 & 0 & 0 & 0\\
    0 & 0 & -1 & 0
\end{array}
\right)
\sim
\begin{equation}
\left(
\begin{array}{c c c | c}
    1 & 1 & 2 & 0\\
    0 & 0 & -1 & 0\\
    0 & 0 & 0 & 0
\end{array}
\right)
\end{equation}
Matice má určitě více než jedno řešení(má jeden sloupec nebazický).
\\
Pro $a \in R \setminus \{1,2\}$, dostávé řešení: \begin{pmatrix} 0\\ 0\\ 0 \end{pmatrix}
Tedy vektory jsou LN pro $a \in R \setminus \{1,2\}$



\end{document}
