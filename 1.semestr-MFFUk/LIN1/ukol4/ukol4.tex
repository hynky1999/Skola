\documentclass[a4paper]{article}

\usepackage[utf8]{inputenc}
\usepackage[T1]{fontenc}
\usepackage{textcomp}
\usepackage[czech]{babel}
\usepackage{amsmath, amssymb}


% figure support
\usepackage{import}
\usepackage{xifthen}
\pdfminorversion=7
\usepackage{pdfpages}
\usepackage{transparent}
\newcommand{\incfig}[1]{%
    \def\svgwidth{\columnwidth}
    \import{./figures/}{#1.pdf_tex}
}

\pdfsuppresswarningpagegroup=1
\author{Hynek Kydlicek}
\title{}
\begin{document}
\maketitle
\section{Dcv. 2}
\begin{align*}
\left(
\begin{array}{ c c c | c c c }
    1 & -1 & 1 & 1 & 0 & 0\\
    0 & 2 & -1 & 0 & 1 & 0\\
    -2 & 0 & 1 & 0 & 0 & 1
\end{array}
\right)
&\sim
\left(
\begin{array}{ c c c | c c c }
    1 & -1 & 1 & 1 & 0 & 0\\
    0 & 2 & -1 & 0 & 1 & 0\\
    0 & -2 & 3 & 2 & 0 & 1
\end{array}
\right)
&\sim
\left(
\begin{array}{ c c c | c c c }
    1 & -1 & 1 & 1 & 0 & 0\\
    0 & 2 & -1 & 0 & 1 & 0\\
    0 & 0 & 2 & 2 & 1 & 1
\end{array}
\right)
&\sim
\\
\left(
\begin{array}{ c c c | c c c }
    1 & -1 & 0 & 0 & -\frac{1}{2} & -\frac{1}{2}\\
    0 & 2 & 0 & 1 & \frac{3}{2} & \frac{1}{2}\\
    0 & 0 & 1 & 1 & \frac{1}{2} & \frac{1}{2}
\end{array}
\right)
&\sim
\left(
\begin{array}{ c c c | c c c }
    1 & 0 & 0 & \frac{1}{2} & \frac{1}{4} & -\frac{1}{4}\\
    0 & 1 & 0 & \frac{1}{2} & \frac{3}{4} & \frac{1}{4}\\
    0 & 0 & 1 & 1 & \frac{1}{2} & \frac{1}{2}
\end{array}
\right)
\end{align*}
\\
Matice 
$
\left(
\begin{array}{ c c c }
     \frac{1}{2} & \frac{1}{4} & -\frac{1}{4}\\
     \frac{1}{2} & \frac{3}{4} & \frac{1}{4}\\
     1 & \frac{1}{2} & \frac{1}{2}
\end{array}
\right)
$
je hledaná inverz.
Oveříme zkouškou.dasfasdf
\begin{align*}
\left(
\begin{array}{ c c c }
    1 & -1 & 1\\
    0 & 2 & -1 \\
    -2 & 0 & 1
\end{array}
\right)
*
\left(
\begin{array}{ c c c }
     \frac{1}{2} & \frac{1}{4} & -\frac{1}{4}\\
    \frac{1}{2} & \frac{3}{4} & \frac{1}{4}\\
    1 & \frac{1}{2} & \frac{1}{2}
\end{array}
\right)
=
\left(
\begin{array}{c c c}
    \frac{1}{2} - \frac{1}{2} + 1 & \frac{1}{4} - \frac{3}{4} + \frac{1}{2} & -\frac{1}{4} -\frac{1}{4} + \frac{1}{2}\\
    1 + -1 & \frac{3}{2} - \frac{1}{2} & \frac{1}{2} - \frac{1}{2}\\
    -1 + 1 & -\frac{1}{2} + \frac{1}{2} & \frac{1}{2} + \frac{1}{2}
\end{array}
\right)
=
\left(
    \begin{array}{c c c}
        1 & 0 & 0\\
        0 & 1 & 0\\
        0 & 0 & 1
    \end{array}
\right)
\end{align*}
\section{Dcv. 3}
\subsection{a)}
Neplatí pokud A = 
$
\begin{bmatrix}
    1 & 0 \\
    0 & 1
\end{bmatrix}
$,$ B =
\begin{bmatrix}
    -1 & 0 \\
    0 & -1
\end{bmatrix}
$,$ A+B=
\begin{bmatrix}
    0 & 0\\
    0 & 0
\end{bmatrix}$
.Matice A,B jsou regulární, ale A+B není.
\subsection{b)}
Z definice symetrické matice $A=A^{T}$.
A je regulární existuje tedy $A^{-1}$.
Proto platí:
\begin{equation}
    A*A^{-1}=I_n
\end{equation}
Transpozice nemění rovnost(pouze přeuspořádáme prvky).
Tranponujeme tedy rovnost.
\begin{align*}
    (A*A^{-1})^T&=(I_n)^T\\
    (A^{-1})^T * A^T &=I_n \text{| viz přednáška}\\
    (A^{-1})^T * A &=I_n \text{| symetrie}
\end{align*}
To znammená, že $(A^{-1})^T$ je inverzní k $A$.
Protože inverze k regularní matici existuje právě jedna, platí $(A^{-1})^T = A^{-1}$.
Tedy inverzní matice $A^{-1}$ je symetrická.
\end{document}
