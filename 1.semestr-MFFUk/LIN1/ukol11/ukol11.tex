\documentclass[a4paper]{article}

\usepackage[utf8]{inputenc}
\usepackage[T1]{fontenc}
\usepackage{textcomp}
\usepackage[czech]{babel}
\usepackage{amsmath, amssymb}


% figure support
\usepackage{import}
\usepackage{xifthen}
\pdfminorversion=7
\usepackage{pdfpages}
\usepackage{transparent}
\newcommand{\incfig}[1]{%
    \def\svgwidth{\columnwidth}
    \import{./figures/}{#1.pdf_tex}
}

\pdfsuppresswarningpagegroup=1

\begin{document}
    \begin{section}{Dcv. 1}
        Ověříme z definice:
        \begin{subsection}{Součet vektorů($f(x) + f(y) = f(x+y)$}
            \[
                x = (a,b), y = (c, d)
            .\] 
            a)
            \begin{align*}
                f(x+y) &=  f(a+c, b+d) = (a + c + 4b + 4d, 2a + 2c - b - d)\\ &= (a + 4b, 2a - b) + (c + 4d, 2c - d) = f(x) + f(y)
            .\end{align*}
            b)
            \begin{align*}
                f(x+y) &=  f(a+c, b+d) = ((a+c)*(b+d), 2b + 2d - a - c)\\ &= (ab + ad + cb + cd, 2b + 2d - a -c) \\ &= (ab, 2b -a) + (cd, 2d - c) + (ad + cd, 0) = f(x) + f(y) + (ad + cd, 0)
            .\end{align*}
            Zobrazení b) podmínku nesplňuje, protipřílad budiž $x = (1,1), y = (1,1)$.
            \[
                f(1,1) + f(1,1) = (2, 2) \neq (4,2) = f(2,2) = f((1,1) + (1,1))
            .\] 
            \begin{subsection}{Násobení vektorů($f(a*x)= a*f(x)$)}
            \[
                x = (a,b)
            .\] 
            a)
            \begin{align*}
                f(u*x) &= f(u*a, u*b) =  (u*a + u*4b, u*2a - u*b) = u*(a + 4b, 2a - b) = f(u)
            .\end{align*}
            Zobrazení a) splňuje obě podmínky, proto je lineární.
            Zobrazení b) nesplňuje první, lineární není.
        \end{subsection}
    \end{section}
    \begin{section}{Dcv. 2}
        Pouze a) je lineární, najdeme jeho matici a obraz.
        Zobrazíme si kanonickou bázi.
        \[
            f(1,0) = (1, 2),
            f(0,1) = (4, -1)
        .\] 
        Jelikož souřadnice v kanonické bázi korespondují se samotnými vektory,
        matice vzhledem ke kanonické bázi vypadá následovně:
        \[
            \begin{pmatrix} 1 && 4 \\ 2 && -1 \end{pmatrix} 
        .\]
        Nyní stačí vynásobit vektor $\begin{pmatrix} 2\\ 1 \end{pmatrix}$, abychom dostali jeho obraz:
        \[
            \begin{pmatrix} 1 && 4 \\ 2 && -1 \end{pmatrix} * \begin{pmatrix} 2\\ 1 \end{pmatrix} = \begin{pmatrix} 6 \\ 3 \end{pmatrix} 
        .\] 
        
    \end{section}
\end{document}
