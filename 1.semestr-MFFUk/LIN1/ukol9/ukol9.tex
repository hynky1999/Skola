\documentclass[a4paper]{article}

\usepackage[utf8]{inputenc}
\usepackage[T1]{fontenc}
\usepackage{textcomp}
\usepackage[czech]{babel}
\usepackage{amsmath, amssymb}


% figure support
\usepackage{import}
\usepackage{xifthen}
\pdfminorversion=7
\usepackage{pdfpages}
\usepackage{transparent}
\newcommand{\incfig}[1]{%
    \def\svgwidth{\columnwidth}
    \import{./figures/}{#1.pdf_tex}
}

\pdfsuppresswarningpagegroup=1
\title{Lin 9}
\author{Hynek Kydlicek}
\begin{document}
\maketitle
\begin{section}{Dcv. 1}
    Zjistíme, zda lze vektor
    $\begin{pmatrix} 4 \\ 4 \\ 2 \end{pmatrix}$ zapsat jako linearní kombinaci vektorů z báze B.
\begin{align*}
    \begin{pmatrix}
        3 & 6 & \bigm| & 4\\
        2 & 1 & \bigm| & 4\\
        4 & 5 & \bigm| & 2
    \end{pmatrix} 
    \sim
    \begin{pmatrix}
        3 & 6 & \bigm| & 4\\
        0 & 4 & \bigm| & 6\\
        0 & 4 & \bigm| & 6
    \end{pmatrix} 
    \sim
    \begin{pmatrix}
        3 & 6 & \bigm| & 4\\
        0 & 4 & \bigm| & 6\\
        0 & 0 & \bigm| & 0
    \end{pmatrix} 
    \sim
    \begin{pmatrix}
        3 & 0 & \bigm| & 2\\
        0 & 4 & \bigm| & 6\\
        0 & 0 & \bigm| & 0
    \end{pmatrix} 
.\end{align*}
Z matice vidíme, že rovnice má právě jedno řešení $\begin{pmatrix}3 \\ 5 \end{pmatrix}$.
Toto řešení jsou zároveň souřadnice vektoru v bázi $B$, což můžeme ověřit:
\[
    \begin{pmatrix}
        3 & 6\\
        2 & 1\\
        4 & 5
    \end{pmatrix}
    \times
    \begin{pmatrix} 3 \\ 5 \end{pmatrix}
    =
    \begin{pmatrix}
        4 \\
        4 \\
        2
    \end{pmatrix} 
.\] 
\end{section}
\begin{section}{Dcv. 2}
    Úlohu můžeme převést na nalezení dimenze prostoru vektorů $T^{4}$, kde jednotlivé složky vektoru(a,b,c,d) jsou koeficienty polynomu a zároveň platí, že $a+b+c+d = 0$.
    Stačí tedy zjisti pro jaké $a,b,c,d \in T$ rovnost platí.
    Tedy stačí zjisti, kolik má rovnice $a+b+c+d = 0$ řešení. Tedy jaká je dimenze Ker(A), \[
        A = \begin{pmatrix} 1 &  1 & 1 & 0 \end{pmatrix} 
    .\] 
    Z Frobienovy věty dostáváme, že dimenze jádra je $n-k$, kde $k$ je počet bazických sloupců.
    Tedy dimenze je 3.
    Nejsem si jist, zda je Frobienova věta již dokázána, proto naleznu bázi o velikost 3.
    Z matice $(A|0)$ dostáváme řešení: \[
        \begin{pmatrix} -x - y \\ x \\ y \\ z \end{pmatrix}, x \in T, y \in T, z \in T
    .\]
    lze zapsat jako
    \[
        x \times \begin{pmatrix} -1 \\ 1 \\ 0 \\ 0 \end{pmatrix}
        + y \times \begin{pmatrix} -1 \\ 0 \\ 1 \\ 0 \end{pmatrix} 
        + z \times \begin{pmatrix} 0 \\ 0 \\ 0 \\ 1 \end{pmatrix} 
    .\] 
    Tedy vidíme, že jádro matice lze zapsat jako množinu všech linéarní kombinací vektorů
        $\begin{pmatrix} -1 \\ 1 \\ 0 \\ 0 \end{pmatrix},
        \begin{pmatrix} -1 \\ 0 \\ 1 \\ 0 \end{pmatrix},
        \begin{pmatrix} 0 \\ 0 \\ 0 \\ 1 \end{pmatrix},$
        \\
        Tyto vektoru tedy generují prostor ker(A), že jde o LN vektory můžeme nahlédnout z definice, popřípadě okometricky :).
\end{section}
\begin{section}{Dcv. 3}
\begin{subsection}{a) podprostor}
    Ověříme 3 podmínky býti podprostorem, neboť S je zajisté podmnožina M.
    \begin{enumerate}
        \item $\theta \in S$\\
            Triviálně z definice je i matice $\begin{pmatrix} 0 & 0 \\ 0 & 0 \end{pmatrix}$ symetrická,
            tedy je opravdu $S$ obsahuje $\theta$.
        \item $x,y \in S \implies x+y \in S$\\
            Tedy máme ukázat, že součet symetrických matic $2\times2$ je sym. matice.\\
            \[
                (x+y)_i_j = x_i_j+y_i_j =(\text{sym. matice}) x_j_i + y_j_i = (x + y)_j_i
            .\] 
        \item $x \in S, \alpha \in T \implies \alpha y \in S$\\
            Tedy máme ukázat, že násobení skalárem zachováva vlastnot být symetrická matice.\\
            \[
                (\alpha x)_i_j = \alpha x_i_j =(\text{sym. matice}) \alpha x_j_i = (\alpha x)_j_i
            .\] 
    \end{enumerate}
    Tedy se skutečně o podprostor jedná.
\end{subsection}
\begin{subsection}{b) Báze}
    Podobně jako v příkladu 2, najdeme všechny matice, které splnǔjí podmínku být symetrické.
    Tedy hledáme matice tvaru $\begin{pmatrix} a & b \\ c & d \end{pmatrix}, a,b,c,d \in T \wedge b =c, \text{lze zapsat jako}\\ b-c = 0$.
    Znovu hledáme jádro matice $A$, kde $A =$
    \[
        \begin{pmatrix} 0 & 1 & -1 & 0 \end{pmatrix}
    .\] 
    Řešení matice $(A|0)$ je:
    \[
        \begin{pmatrix} x \\ y \\ y \\ z  \end{pmatrix} x,y,z \in T
    .\] 
    Tedy jádro matice lze napsat ve tvaru \[
        x \times \begin{pmatrix} 1 \\ 0 \\ 0 \\ 0 \end{pmatrix}
        + y \times \begin{pmatrix} 0 \\ 1 \\ 1 \\ 0 \end{pmatrix} 
        + z \times \begin{pmatrix} 0 \\ 0 \\ 0 \\ 1 \end{pmatrix} 
    .\] 
    Vektory jsou zajisté LN(první vektor má jako jediný nenulový první index, poslední má jako nenulový poslední index, prostřední zbytek) a generují ker(A).
    Nyní stačí vektory bází ker(A) přepsat zpět do hledaného maticového tvaru.
    Tedy báze prostoru symetrických matic jsou matice
    \[
        \begin{pmatrix} 1 & 0 \\ 0 & 0 \end{pmatrix},
        \begin{pmatrix} 0 & 1 \\ 1 & 0 \end{pmatrix},
        \begin{pmatrix} 0 & 0 \\ 0 & 1 \end{pmatrix},
    \]
\end{subsection}
\begin{subsection}{c) dimzenze}
    Jelikož jsme našli bázi o 3 vektorech, tedy je dimenze 3.
\end{subsection}
\end{section}
\end{document}
