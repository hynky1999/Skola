\documentclass{article}
\usepackage{amsmath}
\begin{document}
\section*{Příklad 1}%
\label{sec:Dcv. 1}
\begin{enumerate}
    \item{$4A$} \\[15]
    \begin{pmatrix}
        4 & 8 & 0 & -4\\
        0 & 1 & 4 & 4\\
        8 & 4 & -8 & 16
    \end{pmatrix}
\item{$A+B$} \\[15]
    \begin{pmatrix}
        3 & 6 & 1 & 
    \end{pmatrix}
\end{enumerate}


\section*{Příklad 2}%
\label{sub:Příklad}
Na začatek je dobré říci, že $A^{m,n_1} * B^{n_2,l} = C^{m,l}$, pokud $n_1 \ne n_2$ součin není definován.
Matice je možno sčítát pouze pokud mají stejné rozměry.\\[4]
$B \in \mathbb{R}^{k,.}$, aby byl vůbec mezi maticemi $A$ a $B$ definován součin.\\
$B \in \mathbb{R}^{k,l}$, aby pro výsledný součin $A*B$ byl definován součet s $C$.\\
$A \in \mathbb{R}^{m,k}$, aby byl mezi výrazem $(A*B + C)$ = $\mathbb{R}^{A_{#řádků},l}$ a maticí $E$ definován součin.\\
$C \in \mathbb{R}^{m,l}$, viz řádek nad.\\
$E \in \mathbb{R}^{n,m}$, aby byl mezi výrazem $E*(A*B + C)$ = $\mathbb{R}^{E_{#řádků},l}$ a maticí $D$ definován součet.\\
$D \in \mathbb{R}^{n,l}$, viz řádek nad.\\
$F \in \mathbb{R}^{n,l}$, protože matice $F$ musí mít stejné rozměry jako $D$(Součet matic nemění jejich rozměry).

\section*{Příklad 3}%
\label{sec:Příklad}
$A \in \mathbb{R}^{m,n}$ A_{i,j}, kde $i \in m$ a $j \in n$
\end{document}
